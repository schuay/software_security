\documentclass[a4paper,10pt]{article}

\usepackage[usenames,dvipsnames]{color}
\usepackage{comment}
\usepackage[utf8]{inputenc}
\usepackage{listings}
\usepackage{hyperref}

\definecolor{OliveGreen}{cmyk}{0.64,0,0.95,0.40}
\definecolor{Gray}{gray}{0.5}

\lstset{
    language=C,
    basicstyle=\ttfamily,
    keywordstyle=\color{OliveGreen},
    commentstyle=\color{Gray},
    captionpos=b,
    breaklines=true,
    breakatwhitespace=false,
    showspaces=false,
    showtabs=false,
    numbers=left,
}

\title{VU Software Security \\
       SS 2013 \\
       Task 2}
\author{Jakob Gruber}

\begin{document}

\maketitle

\section{Specification}

Business Worldwide Solutions is a business oriented newspaper focusing on
business news. They recently switched their web presence to an in-house CMS.
The implementation is early beta, but management decided to go live anyways. It
took only a few days until they were HACKED! The attackers seemed to be some
sort of animal activists, at least the CEO thinks so, due to all the pony
pictures that they left on the website. You were called in to help analyse what
happened, and to fix the implementation so that incidents like these can't
happen anymore. The only thing they give you, despite plenty of pressure that
the system has to be back online soon, a copy of the machine as VM - the
username is "softwaresec", the password is "password". Steps to do:

\begin{itemize}
\item Create a timeline of the events. You don't need to reconstruct dates, but
  rather try to figure out the steps the attackers took in order to bring the
  site into the current state. 
\item Explain if the attackers could have gone beyond defacing the website and what
  they might have managed to do. 
\item Categorize the vulnerabilities according to the OWASP Top Ten. (These may
  also give you hints, in case you have missed some vulnerabilities.) 
\item Fix the web app, so all holes are closed. Try to keep your solution as simple
  as possible and explain what you did \& where to fix the holes.
\end{itemize}

Please upload a PDF with your report. For multiple files (e.g., diffs) please
use ZIP. 

\section{Protocol}

This protocol was written in the assumption that the traces in \verb|.bash_history|
and \verb|auth.log| were created during the setup of the virtual machine, and are
not related to the exploit.

\subsection{Event Timeline}

\begin{enumerate}
\item \verb|login.php|

The attack probably started off with a simple SQL injection exploit on the application's
\verb|login.php|. Since the login page is not linked to the initial \verb|index.php|,
the attacker probably guessed the location of the page. \verb|/var/log/apache2/error.log|
does not show any other access attempts, so the attacker seems to have found the correct
page on the first try.

\item SQL injection

Once there, the attacker gained access using an SQL injection vulnerability.
Since the user input is passed straight to the database, it is possible to
trick the system to grant access to any user. This login page does not even mask
SQL error messages, further simplifying the exploit. We were able to gain access
by using the password \verb|' or '' = '|. Note that it is possible to use any arbitrary
user to log in.

\item Shell \& more

The attacker now had several opportunities. In \verb|profile.php|, he could change
the profile information for any user.

\verb|designCheck.php| can be used to read
any file on the system which is readable by the web server, or trick the server into
accessing any desired web page. The information gained by reading logs, config files,
and source files of the web application can help provide further attack vectors.
For example, \verb|designCheck.php?sample=/var/www/backend.php| leads to \verb|includes/businessPage.inc.php|,
which in turn provides the root mysql credentials.

In \verb|backend.php| new articles can be added.
These articles could contain malicious scripts which are executed by visitors' browsers.
The external reference can link to any site. And finally, the any arbitrary file
can be uploaded to the server. The attacker used the latter functionality to upload
\verb|uploads/shell.php|, gaining a shell on the web server running under apache's credentials.

As can be seen in \verb|/var/log/apache2/access.log|, the attacker used this to determine
the system's uptime.

Several articles were created through the admin web interface.

Apache's \verb|error.log| also shows an attempted access to \verb|/etc/passwd|.

\end{enumerate}

\subsection{Further Exploit Scenarios}

Besides defacing the web application, there are several further possible exploit scenarios:

The attacker has full control of all files the apache user can access. For example,
the entire web application could simply be deleted, or arbitrary new web sites could
be uploaded and made accessible online.

Since the database credentials were located in a readable php file, the attacker
has complete access to the mysql database.

As mentioned, most config files are available to the intruder; he could check these
for opportunities for further attacks. For example, \verb|/etc/passwd| could
be used to harvest available user names for ssh attacks. Maybe some user will
even have the same password as the database root user?

The server could be tricked into accessing any url through \verb|designCheck.php|.
This could be used to incriminate the owner with illegal content.

Session cookies are also world-readable and can be stolen.

A logged in user could be exploited to change his account details or post new articles through
cross site request forgery.

Images posted in authentic articles could be overwritten and changed by any logged in user.

\subsection{Categorization of Vulnerabilities}

The following vulnerabilities are categorized according to the OWASP top 10 2013RC1.

\begin{itemize}
\item SQL injection: A1, A2
\item Shell access through the weather form: A1
\item Comment creation can be enabled by setting comments=1: A2
\item Images can be overwritten by all logged in users: A2
\item Scripts in articles and article comments: A3
\item Access to files through \verb|designCheck.php|: A4
\item Outdated versions of apache, mysql, php: A5
\item Enabled ssh root login: A5
\item Directory browsing is enabled in apache: A5
\item Plain text passwords stored in database files: A6
\item Access to \verb|designCheck.php| possible without login: A7
\item No user separation in profile and article pages: A7
\item Another site could exploit a user who is currently logged into this web application
      through cross site request forgery. For example the user could be made to
      change his profile without his knowledge or consent: A8
\item PHP5 (and probably other components of the stack as well) has known security vulnerabilities,
      even when updated to the latest repository version. See \url{http://www.cvedetails.com/vulnerability-list/vendor_id-74/product_id-128/version_id-125887/PHP-PHP-5.3.10.html}: A9
\item \verb|redirect.php| does not validate the target: A10
\end{itemize}

\subsection{Securing the Application}

Note: meticulous error handling was not done in the course of this task to keep the work load
manageable. A real-world application would need more work in this area.

\begin{itemize}
\item Stricter sshd configuration. Disallowing root logins and specifically
      set allowed users makes ssh attacks more difficult.

\begin{verbatim}
--- /etc/ssh/sshd_config        2013-03-04 21:06:21.137976288 +0100
+++ /etc/ssh/sshd_config.new    2013-04-11 18:16:56.079223014 +0200
@@ -22,9 +22,11 @@
 SyslogFacility AUTH
 LogLevel INFO
 
+AllowUsers softwaresec
+
 # Authentication:
 LoginGraceTime 120
-PermitRootLogin yes
+PermitRootLogin no
 StrictModes yes
 
 RSAAuthentication yes
\end{verbatim}

\item Update the system. Critical system components should always be up-to-date
      to ensure we can benefit from the most recent security fixes.

\begin{verbatim}
$ sudo apt-get update
$ sudo apt-get upgrade
\end{verbatim}

\item Set a decent password for the user softwaresec. Strong passwords are long,
      consist of a combination of alphanumeric and punctuation symbols, and avoid
      common words or variations thereof.

\begin{verbatim}
$ passwd
\end{verbatim}

\item Remove \verb|designCheck.php|. The functionality is inherently unsafe. We
      remove the file from the physical filesystem and from all related files
      in the web application.
      
\item Remove all traces of the attack. This includes all files in \verb|/var/www/uploads|
      and the articles as well as the users added by the attacker. The safest way
      to do this would be to restore the database and all application files from 
      a known safe backup.
      
\item Secure the apache configuration. This is of course only a partial
      configuration and much more could be done to set up apache securely. However,
      this is a good start. We disable the server signature, disallow access to directories
      outside of \verb|/var/www|, and disable directory listings.

\begin{verbatim}
--- /etc/apache2/sites-enabled/000-default.old  2013-04-11 19:49:41.467016818 +0200
+++ /etc/apache2/sites-enabled/000-default      2013-04-11 19:52:31.183009546 +0200
@@ -1,13 +1,18 @@
+ServerSignature Off
+ServerTokens Prod
+
 <VirtualHost *:80>
        ServerAdmin webmaster@localhost
 
        DocumentRoot /var/www
        <Directory />
-               Options FollowSymLinks
+               Order Deny,Allow
+               Deny from all
+               Options none
                AllowOverride None
        </Directory>
        <Directory /var/www/>
-               Options Indexes FollowSymLinks MultiViews
+               Options FollowSymLinks MultiViews
                AllowOverride None
                Order allow,deny
                allow from all

\end{verbatim}

\item Ensure all user interactions with the database are done through prepared statements.
      This was actually quite a lot of work, since every mysql query had to be adapted. The new usage
      pattern is always similar to the following snippet (full diffs are also available separately).
      
\lstset{language=php}
\begin{lstlisting}
        $mysqli = databaseConnect();         
        $stmt = $mysqli->prepare("INSERT INTO comments (comment,user,articleid) VALUES (?, ?, ?);");
        $stmt->bind_param("ssi", $message, $name, $id);
        $stmt->execute();
        $stmt->close();
        $mysqli->close();
\end{lstlisting}

\item Do not display database error messages on the web application. These can help an attacker
      to gain access. All such error displays have been removed from the application.

\item Prevent changing another user's profile by grabbing the currently logged in user from
      the business session instead of the id parameter. New profile creation has been disabled as well (these can be added directly to the database).
      As always, details are available in the full diff files.

\item Restrict file uploads to images of type jpg, jpeg, gif, and png. File sizes are already
      restricted.
      
\item Prevent overwriting other uploads by choosing a unique name for the uploaded file.

\item Prevent the built-in shell access through the weather query by moving the user input
      out of the \lstinline|shell_exec()| call.

\item Prevent XSS in comments and similar fields by wrapping all of these inputs with \lstinline|htmlspecialchars|.

\item Separate database users for each web application. Currently, there is only a single application,
      but in the future there might be new additions. We created a new mysql user, granted the appropriate
      rights to all tables, and changed the credentials hardcoded in the php files.
      
\begin{verbatim}
mysql> create user 'bws'@'localhost' identified by 'Y_pwQZF2367w';
mysql> grant all privileges on business.* to 'bws'@'localhost';
\end{verbatim}

\item Encrypt user passwords in the database using \verb|phpass|. These hashes are salted to
      prevent rainbow table attacks. The column \verb|password| in the \verb|users| table
      had to be modified to \verb|varchar(128)| in order to have enough capacity for
      the complete hash. Further details are in the diff files.
      
\item Hide password as the user is entering it by using the input type ``password''.

\item Prevent cross-site request forgeries by creating a session token and validating
      user input against it. All invalid requests are aborted. % TODO: Protection of our areas enough? Comments? Logout?

\item Disable redirects to unverified sites. Redirects targets are verified to point
      to a trusted location, and ignored otherwise. This removes functionality such as redirecting
      to a specific page after login which would need to be reimplemented using secure
      mechanisms.

\item Prevent setting article authors to another user by always defaulting to the logged
      in user.

\end{itemize}

\end{document}
