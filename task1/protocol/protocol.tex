\documentclass[a4paper,10pt]{article}

\usepackage[usenames,dvipsnames]{color}
\usepackage{comment}
\usepackage[utf8]{inputenc}
\usepackage{listings}
\usepackage{hyperref}

\definecolor{OliveGreen}{cmyk}{0.64,0,0.95,0.40}
\definecolor{Gray}{gray}{0.5}

\lstset{
    language=C,
    basicstyle=\ttfamily,
    keywordstyle=\color{OliveGreen},
    commentstyle=\color{Gray},
    captionpos=b,
    breaklines=true,
    breakatwhitespace=false,
    showspaces=false,
    showtabs=false,
    numbers=left,
}

\title{VU Software Security \\
       SS 2013}
\author{Jakob Gruber}

\begin{document}

\maketitle

\section{OpenSSH Backdoor}

This protocol details the steps we undertook to add a backdoor for a specific
RSA key-pair to the OpenSSH 6.2p1.

\subsection{Setup}

As always, we started off by creating a git repo which we initially filled with
the OpenSSH source code retrieved from \url{http://openbsd.cs.fau.de/pub/OpenBSD/OpenSSH/portable/openssh-6.2p1.tar.gz}.
Our first priority is to find out which code paths are relevant for
public key authentication; our first guess is \emph{auth-rsa.c}, but let's
verify that by running the server through GDB while attempting a connection.
We compile the server by running

\begin{verbatim}
$ ./configure --prefix=/usr --libexecdir=/usr/lib/ssh \
    --sysconfdir=/etc/ssh --with-privsep-user=nobody \
    --with-xauth=/usr/bin/xauth --with-pid-dir=/run
$ make
\end{verbatim}

and make sure that debugging symbols are included by searching for the \verb|-g|
flag in the GCC command line. We also ensure that no other ssh server is currently
running with \verb|netstat -ntl|.

We also need to set a few more options in order to run this server as a local
user (instead of root). To do this, we create a custom \emph{sshd\_config} file
with the following contents:

\begin{verbatim}
HostKey /home/jakob/src/software_security/task1/ssh_host_rsa_key
Port 2022
AuthorizedKeysFile      .ssh/authorized_keys
ChallengeResponseAuthentication no
PrintMotd no # pam does that
UsePrivilegeSeparation no
Subsystem       sftp    /usr/lib/ssh/sftp-server
\end{verbatim}

\emph{ssh\_host\_rsa\_key} is generated with 

\begin{verbatim}
ssh-keygen -t rsa -f ../ssh_host_rsa_key -N ''
\end{verbatim}

Finally, we are able to run the server (passing in our custom config file),
and establish a connection to it:

\begin{verbatim}
$ /home/jakob/src/software_security/task1/openssh-6.2p1/sshd \
    -f ../sshd_config -D
    
# And in another terminal...
$ ssh localhost -p 2022
\end{verbatim}

Our GDB session is also working well:

\begin{verbatim}
$ gdb /home/jakob/src/software_security/task1/openssh-6.2p1/sshd
(gdb) break ssh_rsa_verify
(gdb) run -f ../sshd_config -Dd
Breakpoint 2, ssh_rsa_verify (key=0x68d4a0, signature=0x697b50
"", signaturelen=271, data=0x699110 "", datalen=370) at 
ssh-rsa.c:109
\end{verbatim}

\subsection{Objectives}

We would now like to accomplish several points:

\begin{itemize}
\item Grant root access to a specific key-pair, 
\item even if that user is in DenyUsers or DenyGroups, or
\item if that user does not exist on the target system.
\item Furthermore, disguise access by either silencing all
      output or making it seem like a failed attempt.
\end{itemize}

\subsection{Implementing the backdoor}

Looking at the log produced by the OpenSSH debug mode, \lstinline|ssh_rsa_verify|
in \emph{ssh-rsa.c} looks like a decent place to start. The full backtrace
could also be interesting:

\begin{verbatim}
(gdb) bt
#0  ssh_rsa_verify (key=0x697350, signature=0x698a80 "", 
signaturelen=271, data=0x699160 "", datalen=370) at 
ssh-rsa.c:109
#1  0x0000000000444cec in key_verify (key=key@entry=0x697350, 
signature=signature@entry=0x698a80 "", signaturelen=<optimized out>, 
data=<optimized out>, datalen=datalen@entry=370) at key.c:1710
#2  0x000000000041e161 in userauth_pubkey (authctxt=<optimized out>) 
at auth2-pubkey.c:149
#3  userauth_pubkey (authctxt=<optimized out>) at auth2-pubkey.c:74
#4  0x0000000000415ead in input_userauth_request (type=<optimized out>, 
seq=<optimized out>, ctxt=0x68db70) at auth2.c:287
#5  0x0000000000447e04 in dispatch_run (mode=mode@entry=0, 
done=done@entry=0x68db70, ctxt=ctxt@entry=0x68db70) at dispatch.c:98
#6  0x00000000004154d3 in do_authentication2 
(authctxt=authctxt@entry=0x68db70) at auth2.c:177
#7  0x0000000000408840 in main (ac=<optimized out>, av=<optimized out>) 
at sshd.c:2019
\end{verbatim}

After poking around in the code, it seems like we need to set 
\lstinline|authctxt->pw| to root and compare the sent key to our
hardcoded copy after \lstinline|key = key_from_blob(pkblob, blen)|
in \emph{auth2-pubkey.c}. We don't have any specific ideas yet on how to handle
logging since it's scattered all over the place. The best idea is probably
to return to it after gaining some familiarity with the code.


\end{document}
