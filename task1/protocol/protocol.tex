\documentclass[a4paper,10pt]{article}

\usepackage[usenames,dvipsnames]{color}
\usepackage{comment}
\usepackage[utf8]{inputenc}
\usepackage{listings}
\usepackage{hyperref}

\definecolor{OliveGreen}{cmyk}{0.64,0,0.95,0.40}
\definecolor{Gray}{gray}{0.5}

\lstset{
    language=C,
    basicstyle=\ttfamily,
    keywordstyle=\color{OliveGreen},
    commentstyle=\color{Gray},
    captionpos=b,
    breaklines=true,
    breakatwhitespace=false,
    showspaces=false,
    showtabs=false,
    numbers=left,
}

\title{VU Software Security \\
       SS 2013}
\author{Jakob Gruber}

\begin{document}

\maketitle

\section{OpenSSH Backdoor}

This protocol details the steps we undertook to add a backdoor for a specific
RSA key-pair to the OpenSSH 6.2p1.

\subsection{Setup}

As always, we started off by creating a git repo which we initially filled with
the OpenSSH source code retrieved from \url{http://openbsd.cs.fau.de/pub/OpenBSD/OpenSSH/portable/openssh-6.2p1.tar.gz}.
Our first priority is to find out which code paths are relevant for
public key authentication; our first guess is \emph{auth-rsa.c}, but let's
verify that by running the server through GDB while attempting a connection.
We compile the server by running

\begin{verbatim}
$ ./configure --prefix=/usr --libexecdir=/usr/lib/ssh \
    --sysconfdir=/etc/ssh --with-privsep-user=nobody \
    --with-xauth=/usr/bin/xauth --with-pid-dir=/run \
    CFLAGS="-g -O0"
$ make
\end{verbatim}

and make sure that debugging symbols are included by searching for the \verb|-g|
flag in the GCC command line. We also ensure that no other ssh server is currently
running with \verb|netstat -ntl|.

We also need to set a few more options in order to run this server as a local
user (instead of root). To do this, we create a custom \emph{sshd\_config} file
with the following contents:

\begin{verbatim}
HostKey /home/jakob/src/software_security/task1/ssh_host_rsa_key
Port 2022
AuthorizedKeysFile      .ssh/authorized_keys
ChallengeResponseAuthentication no
PrintMotd no # pam does that
UsePrivilegeSeparation no
Subsystem       sftp    /usr/lib/ssh/sftp-server
\end{verbatim}

\emph{ssh\_host\_rsa\_key} is generated with 

\begin{verbatim}
ssh-keygen -t rsa -f ../ssh_host_rsa_key -N ''
\end{verbatim}

Finally, we are able to run the server (passing in our custom config file),
and establish a connection to it:

\begin{verbatim}
$ /home/jakob/src/software_security/task1/openssh-6.2p1/sshd \
    -f ../sshd_config -D
    
# And in another terminal...
$ ssh localhost -p 2022
\end{verbatim}

Our GDB session is also working well:

\begin{verbatim}
$ gdb /home/jakob/src/software_security/task1/openssh-6.2p1/sshd
(gdb) break ssh_rsa_verify
(gdb) run -f ../sshd_config -Dd
Breakpoint 2, ssh_rsa_verify (key=0x68d4a0, signature=0x697b50
"", signaturelen=271, data=0x699110 "", datalen=370) at 
ssh-rsa.c:109
\end{verbatim}

\subsection{Objectives}

We would now like to accomplish several points:

\begin{itemize}
\item Grant access to a specific key-pair, 
\item Furthermore, disguise access by either silencing all
      output or making it seem like a failed attempt.
\end{itemize}

These objectes will be achieved under the assumption that the target exists
on the target system, that ``publickey'' authentication is enabled, and that
\verb|DenyUsers,DenyGroups,AllowUsers,AllowGroups| are not in use.

\subsection{Implementing the backdoor}

Looking at the log produced by the OpenSSH debug mode, \lstinline|ssh_rsa_verify|
in \emph{ssh-rsa.c} looks like a decent place to start. The full backtrace
could also be interesting:

\begin{verbatim}
(gdb) bt
#0  ssh_rsa_verify (key=0x697350, signature=0x698a80 "", 
signaturelen=271, data=0x699160 "", datalen=370) at 
ssh-rsa.c:109
#1  0x0000000000444cec in key_verify (key=key@entry=0x697350, 
signature=signature@entry=0x698a80 "", signaturelen=<optimized out>, 
data=<optimized out>, datalen=datalen@entry=370) at key.c:1710
#2  0x000000000041e161 in userauth_pubkey (authctxt=<optimized out>) 
at auth2-pubkey.c:149
#3  userauth_pubkey (authctxt=<optimized out>) at auth2-pubkey.c:74
#4  0x0000000000415ead in input_userauth_request (type=<optimized out>, 
seq=<optimized out>, ctxt=0x68db70) at auth2.c:287
#5  0x0000000000447e04 in dispatch_run (mode=mode@entry=0, 
done=done@entry=0x68db70, ctxt=ctxt@entry=0x68db70) at dispatch.c:98
#6  0x00000000004154d3 in do_authentication2 
(authctxt=authctxt@entry=0x68db70) at auth2.c:177
#7  0x0000000000408840 in main (ac=<optimized out>, av=<optimized out>) 
at sshd.c:2019
\end{verbatim}

After poking around in the code, it seems like we need to set 
\lstinline|authctxt->pw| to root and compare the sent key to our
hardcoded copy after \lstinline|key = key_from_blob(pkblob, blen)|
in \emph{auth2-pubkey.c}. We don't have any specific ideas yet on how to handle
logging since it's scattered all over the place. The best idea is probably
to return to it after gaining some familiarity with the code.

We now step through a valid authentication attempt with GDB. The first interesting
bit are the contents of \lstinline|authctxt->pw| after \lstinline|getpwnamallow(user)|:

\begin{verbatim}
(gdb) p *authctxt->pw
$9 = {pw_name = 0x68e6c0 "jakob", pw_passwd = 0x68e9f0 "x",
pw_uid = 1000, pw_gid = 1000, pw_gecos = 0x68ea40 "",
pw_dir = 0x68ea60 "/home/jakob", pw_shell = 0x68ea80 "/bin/bash"}
\end{verbatim}

The first attempt is done with the ``none'' method in \lstinline|userauth_none|.
This obviously fails, but is not counted towards the failed attempt count.
Then, surprisingly, \lstinline|userauth_pubkey| is executed twice. Debug output
from both attempts:

\begin{verbatim}
# The 1st time
debug1: test whether pkalg/pkblob are acceptable
debug1: temporarily_use_uid: 1000/1000 (e=1000/1000)
debug1: trying public key file /home/jakob/.ssh/authorized_keys
debug1: fd 4 clearing O_NONBLOCK
debug1: matching key found: file /home/jakob/.ssh/authorized_keys,
line 2
Found matching RSA key:
46:5b:94:27:1f:62:10:00:75:99:58:9a:ad:e5:b5:c7
debug1: restore_uid: (unprivileged)

# The 2nd time
debug1: temporarily_use_uid: 1000/1000 (e=1000/1000)
debug1: trying public key file /home/jakob/.ssh/authorized_keys
debug1: fd 4 clearing O_NONBLOCK
debug1: matching key found: file /home/jakob/.ssh/authorized_keys,
line 2
Found matching RSA key:
46:5b:94:27:1f:62:10:00:75:99:58:9a:ad:e5:b5:c7
debug1: restore_uid: (unprivileged)
debug1: ssh_rsa_verify: signature correct
\end{verbatim}

The key seems (which we will need to check) seems to be encoded in \lstinline|pkblob|.
If we call \lstinline|dump_base_64(stderr, pkblob, blen)|, our public key is
printed to the console. To get a C representation of our public key, we use
the following commands:

\begin{verbatim}
$ grep -o 'AAAA[^ ]*' ssh_host_rsa_key.pub |
  base64 -d > pubkey.decoded
$ xxd -i pubkey.decoded > openssh-6.2p1/pubkey.h
\end{verbatim}

Including this new header in the code, constructing a key from it and comparing
these with \lstinline|key_equal| returns true, which gives us a way to identify
the correct key.

Next, let's take a look at logging. \emph{log.c} tells us that all logging functions
use \lstinline|do_log|; and to silence logging, all we need to do is introduce
a flag which causes \lstinline|do_log| to return early if it is set.

It is now fairly trivial to implement our backdoor. Once the authentication process
reaches the ``publickey'' method, we simply need to check whether the key matches
our backdoor key. If it does, we set authenticated to \lstinline|true| (which logs
us in through \lstinline|userauth_finish|) and disable logging by setting
\lstinline|log_silenced = true|, which skips all logic in \lstinline|do_log|.

We also disables any updates to lastlog which were triggered by calls to \lstinline|login_login|
and \lstinline|login_logout|, as well as disabling the check whether root logins
were allowed.

This approach has some weaknesses:

\begin{itemize}
\item Logging is only disabled once we can actually determine that the key matches
      the backdoor key. Any prior logs go through unimpeded. Keeping a perfectly
      valid looking log would require much mure work.
\item Only the public key is compared while the private signature is ignored. Anybody
      with the correct public key can log in to the backdoor. While we could wait
      for the second ``publickey'' attempt (which includes the signature), this
      would be more work for little benefit.
\end{itemize}


\end{document}
