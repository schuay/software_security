\documentclass[a4paper,10pt]{article}

\usepackage[usenames,dvipsnames]{color}
\usepackage{comment}
\usepackage[utf8]{inputenc}
\usepackage{listings}
\usepackage[pdfborder={0 0 0}]{hyperref}

\definecolor{OliveGreen}{cmyk}{0.64,0,0.95,0.40}
\definecolor{Gray}{gray}{0.5}

\lstset{
    language=C,
    basicstyle=\ttfamily,
    keywordstyle=\color{OliveGreen},
    commentstyle=\color{Gray},
    captionpos=b,
    breaklines=true,
    breakatwhitespace=false,
    showspaces=false,
    showtabs=false,
    numbers=left,
}

\title{VU Software Security \\
       SS 2013 \\
       Task 3}
\author{Jakob Gruber}

\begin{document}

\maketitle

\section{Specification}

When you try to run the binary, you'll immediately see that is not doing much,
because it is controlled by some kind of trigger condition.

This assignment consists in two parts:

Analyze the binary and reveal the trigger that controls the execution of the
program.  Reverse engineer the check, set the right condition and find the
secret message.

Recommended tools:

Debugger, Hexeditor, ...

Please upload a PDF with your report (you do not have to upload the modified
binary)! 

\section{Protocol}

Our first steps were to find out a little more obvious information about the binary:

\begin{verbatim}
$ file challenge
challenge: ELF 64-bit LSB  executable, x86-64, version 1 (SYSV),
dynamically linked (uses shared libs), for GNU/Linux 2.6.32, stripped
$ strings challenge
[...]
$ strace ./challenge
[...]
$ ./challenge
'arlogh Qoylu'pu'?
\end{verbatim}

Running the binary itself outputs a scrambled string for now.
\verb|strings| showed us all external library calls:

\begin{verbatim}
perror
mprotect
ptrace
printf
sysconf
signal
alarm
gettimeofday
exit
memalign
EVP_CIPHER_CTX_cleanup
EVP_CIPHER_CTX_init
EVP_DecryptFinal_ex
EVP_DecryptInit_ex
EVP_DecryptUpdate
EVP_aes_128_ofb
SHA1
memcpy
sysconf
memalign
mprotect
\end{verbatim}

Just by looking at this list, we can tell a couple of things:

\begin{itemize}
\item \lstinline|ptrace|: The application probably tries to protect against attached debuggers
\item \lstinline|alarm|: A SIGALARM signal is generated at some time
\item \lstinline|gettimeofday|: The current time and date is used in some way
\item \lstinline|EVP_*|: Data is decrypted using a symmetric cipher
\item \lstinline|SHA1|: Something is being hashed
\end{itemize}

In hindsight, it seems fairly obvious that \lstinline|gettimeofday| was going to be important.
If we would have followed the data retrieved by \lstinline|gettimeofday| instead of
reverse engineering the entire binary, we could have saved much time.

Next, we fired up GDB and ran into the \lstinline|ptrace| check, which was easily
disabled by replacing the original \lstinline|JGE| opcode with an unconditional jump
\lstinline|NOP; JMP|.

\begin{lstlisting}
  400dda:	e8 a1 fc ff ff       	callq  400a80 <ptrace@plt>
  400ddf:	48 3d 00 00 00 00    	cmp    $0x0,%rax
  400de5:	90                   	nop
  400de6:	e9 7b 00 00 00       	jmpq   400e66 <mprotect@plt+0x346>
\end{lstlisting}

We then spent several hours reverse engineering the assembly code, building something
resembling a control flow graph, and trying different conditional jump branches by
poking at register values with GDB. We tried things such as removing the \lstinline|alarm|
call, or switching several conditional jumps which lead to \lstinline|printf(); exit();| sections.
Sadly, none of this had any effect other than producing various segfaults.

Our next attack vector was to hook into interesting library function calls and
print their arguments. A snippet of the hook library is reproduced here:

\begin{lstlisting}
unsigned char *SHA1(const unsigned char *d,
                    unsigned long n,
                    unsigned char *md)
{    
	static unsigned char *(*fn)(const unsigned char *, unsigned long, unsigned char *) = NULL;
    if (!fn) {
        fn = dlsym(RTLD_NEXT, "SHA1");
    }
    DEBUG(printf("PRE  SHA1(%s, %d, %s)\n", d, n, md);)
    DEBUG(hexdump("d", d, n);)
    unsigned char *p = fn(d, n, md);
    DEBUG(printf("POST SHA1(%s, %d, %s) = %s\n", d, n, md, p);)
    DEBUG(hexdump("md", md, 20);)
    return p;
}
\end{lstlisting}

The hook may be attached by preloading it:

\begin{verbatim}
$ LD_PRELOAD=$(pwd)/hook.so ./challenge
\end{verbatim}

Again, this did not produce any new results. After a good night's sleep,
we decided to get back to analyzing the disassembly. Two addresses in particular
appeared to be interesting: \verb|0x402b00| and \verb|0x4027d0|. We began
by following accesses to the latter address.

This address is first set to \lstinline_(time & 0xffff0000) | 0x0000c0de)_, and later
hashed by \lstinline|SHA1()|. The result is then used as the key to decrypt some data segment.
Seems like we hit the jackpot! Since only two bytes are used from the time returned by 
\lstinline|gettimeofday|, the search space is rather small and can easily be brute-forced.

We proceeded to do this by instrumenting our hook to override the time returned by \lstinline|gettimeofday| with the value of an environment variable:

\begin{lstlisting}
const char *time = getenv("REAL_TIME");
int p = fn(tp, tzp);

if (time != NULL) {
    int t = strtol(time, NULL, 16);
    int *tpp = tp;
    *tpp = t;
}
\end{lstlisting}

A short groovy script found the correct \lstinline|gettimeofday| value within a minute or two:

\begin{lstlisting}
final workingDir = new File('.').getCanonicalPath()

(0x0000 .. 0xffff).each {
    final time = it * 0x10000
    final proc = "./challenge".execute(["REAL_TIME=${Integer.toHexString(time)}", "LD_PRELOAD=$workingDir/hook.so"], null)

    stdout = proc.in.text
    if (!(stdout =~ /^'arlogh Qoylu'pu'?/)) {
        print "time: ${Integer.toHexString(time)}: $stdout"
    }
}
\end{lstlisting}

\begin{verbatim}
$ groovy Exploit/src/Exploit.groovy 
time: af6a0000: This is not the secret you are looking for!
To dig deeper you have!
\end{verbatim}

Maybe there's some more stuff hidden in the decrypted data. Hooking into \lstinline|EVP_DecryptUpdate|,
we receive:

\begin{verbatim}
POST EVP_DecryptUpdate(717eaaf0, 7bc000: 
The secret is: "The truth is out there!"
This is not the secret you are looking for!
To dig deeper you have!, 160, 402800, a0) = 0x7f6900000001
\end{verbatim}

The secret seems to be: \emph{The truth is out there!}.

The following is our final reconstruction of the relevant parts of the program:

\begin{lstlisting}
0x402b00 := fptr
0x4027d0 := key

from __libc_start_main->__libc_csu_init->initarray[2]:
{
  alarm(1);
  signal(SIGALARM, 0x400db0);

  // sets up 0x4027d0 which is used later as as the key to
  // decryption through SHA1()
  gettimeofday(&time, NULL);
  *0x4027d0 = some_transformation(time);

  // *0x402b00 = memalign(), sets up dynamic memory area
  sysconf();
  *0x402b00 = memalign(4096, 16384);

  mprotect()
}

main {
  if (do_weird_calcs() == 0) {
    print error and exit;
  }
  
  // uses SHA1(key) as the key to decrypt data into fptr.
  // fptr[0..3] = 0x540a6eeb (jmp to fptr + ??)
  // fptr[4..?] = "The secret is: "The truth is out there!"
  //		  This is not the secret you are looking for!
  //		  To dig deeper you have!"
  // fptr[...] = ?
  sha1_and_evp();

  if (fptr[0] == 0x540a6eeb) {
    fptr();
    return;
  }

  ...
}

fptr() {
  write(STDOUT_FILENO, 0x40402c, 0x44); // This is not the secret you are looking for!
					// To dig deeper you have!"
  exit(0);
}
\end{lstlisting}

The following links were our main references during this task:

\begin{itemize}
\item Syscall numbers: \url{http://strace.git.sourceforge.net/git/gitweb.cgi?p=strace/strace;a=blob;f=linux/x86_64/syscallent.h;h=8e3a2007ad7bda8fbc7318fd081c6fff3ed2a369;hb=HEAD}
\item Callings conventions: \url{http://stackoverflow.com/questions/2535989/what-are-the-calling-conventions-for-unix-linux-system-calls-on-x86-64}
\item \lstinline|__libc_start_main| tutorial: \url{http://dbp-consulting.com/tutorials/debugging/linuxProgramStartup.html}
\end{itemize}

The entire annotated disassembly is included as a separate file.

\end{document}
